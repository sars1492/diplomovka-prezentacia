\section{Definícia pojmov}

\subsection{Bibliometria}

\begin{frame}
  \frametitle{Definícia bibliometrie}

Termín bibliometria je zložený z~dvoch gréckych slov:
\textporson{biblion}\index{biblion@\textporson{biblion}}, čo zamená kniha, a
\textporson{métron}\index{métron@\textporson{métron}}, meranie.  Takže doslovný
preklad by bol \uv{meranie kníh}, alebo \uv{veda zaoberajúca sa meraním kníh}.
\citet{Pritchard1969} pôvodne definoval bibliometriu ako aplikácia
matematických a štatistikých metód na knihy a iné komunikačné média.

\end{frame}

\begin{frame}
  \frametitle{Definícia bibliometrie: pokčačovanie}

V~súčasnosti sa pod týmto termínom chápe súhrn štatistickým metód na
kvantitatívnu analýzu publikácií v~písomnej forme, ako sú knihy, alebo články
vo forme bibliografických záznamov.  Tieto záznamy zahrňujú informácie ako
názov publikácie, jej autorov, rok publikovania, ale aj kľúčové slová,
abstrakt, či referencie na iné publikácie (citácie).  Podľa Ondrišovej
\citeyearpar{Ondrisova2011} môžeme na bibliometrických záznamoch študovať:

\begin{itemize}
\item aspekty tvorby publikácií:  zoznam autorov, zoznam použitej literatúry;
\item aspekty šírenia publikácií:  časopis, vydavatel, webové stránky;
\item aspekty použitia publikácií:  citácie, frekvencia  požičiavania, alebo
    prístupu kníh v~knižnici, alebo aj štatistika návštevnosti na webe.

\end{itemize}

\end{frame}

\begin{frame}
  \frametitle{Citačná analýza}

Najčastejšia bibliometrická metóda je tzv.\,citačná analýza,  pri ktorej sa štatisticky spracovávajú citácie.  V~nej sa ďalej
zahrňujú ostatné informácie bibliografických záznamov, ako počet autorov
(priemerný počet autorov na dokument, priemerný počet citácií na autora za
dokument), počet strán (priemerný počet citácií na stranu dokumentu), počet
publikácií v~konkrétnom časopise a zmeny týchto informácií za isté obdobie.  To
znamená, že analýzou dát z~bibliometrických záznamov môžeme sledovať vývoj
jednotlivých oblastí, ich vzájomný vplyv a prepojenia.

\end{frame}

\begin{frame}
  \frametitle{Kocitačná analýza}

\citet{Vavrikova2008} uvádza špecifický druh citačných analýz: tzv.\,kocitačné
analýzy.   Určujú podobnosť medzi dvoma elementy. Ak
elementy A~i B sú citované elementom C, môžeme uvažovať o~ich vzájomnom vzťahu,
aj keď na seba priamo neodkazujú. V~prípade ak elementy A~a B sú citované
viacerými ďalšími elementami, ich vzájomný vzťah je silnejší čím sú spoločne
viac citované. Z~počiatku kocitácie boli navrhnuté ako zakladná metrika na
charakterizáciu podobnosti medzi dokumentami. Teraz sa s~kocitačnou analýzou
môžeme stretnúť pri vyhľadávaní podobných dokumentov v~databázach (related
documents search), ktoré je niekedy nazývané ako \uv{pattern search} tzv.
vzorové vyhľadávanie. Vyhľadávacie algoritmy berú do úvahy spoločných autorov,
ale aj kľúčové slová.

\end{frame}

